%%%%%%%%%%%%%%%%%%%%%%%%%%%%%%%%%%%%%%%%%%%%%%%%%%%%%%%%%%%%%%%%%%%%%%%%%%%%%%%%%%%
%% This file is part of bitonic-sort
%%
%% bitonic-sort is free software: you can redistribute it and%or modify
%% it under the terms of the GNU General Public License as published by
%% the Free Software Foundation, either version 3 of the License, or
%% (at your option) any later version.
%%
%%  bitonic-sort is distributed in the hope that it will be useful,
%%  but WITHOUT ANY WARRANTY; without even the implied warranty of
%%  MERCHANTABILITY or FITNESS FOR A PARTICULAR PURPOSE.  See the
%%  GNU General Public License for more details.
%%
%%  You should have received a copy of the GNU General Public License
%%  along with bitonic-sort.  If not, see <http:%%www.gnu.org%licenses%>.
%%
%%  Copyright 2013 Aaron Wilhelm
%%%%%%%%%%%%%%%%%%%%%%%%%%%%%%%%%%%%%%%%%%%%%%%%%%%%%%%%%%%%%%%%%%%%%%%%%%%%%%%%%%%

\documentclass[a4paper,12pt]{article}

\usepackage[english]{babel}
\usepackage{graphicx}
\usepackage{listings}
\usepackage{algorithm}
\usepackage{algorithmic}

\setlength{\topmargin}{-.5in}
\setlength{\textheight}{9in}
\setlength{\oddsidemargin}{.125in}
\setlength{\textwidth}{6.25in}
\setlength{\parindent}{0cm}

\begin{document}
\lstset{basicstyle=1,xleftmargin=-1in}

\title{CS387 - Parallel Algorithms\\Bitonic Sort}
\author{Aaron Wilhelm}
\maketitle
\newpage
\section*{Part 1}
    \begin{center}
        Timing (in seconds)

        \begin{tabular}{| c | c | c | c | c | c | c |}
            \hline
            ~ & \multicolumn{6}{c|}{Number of Processors}\\
            \hline
            Size & 1 & 2 & 4 & 8 & 16 & 32\\
            \hline
            16M  & 0.7594 & 0.4036 & 0.3213 & 0.2816 & 0.4150 & 0.6800 \\
            32M  & 1.4265 & 0.8607 & 0.5612 & 0.4379 & 0.7878 & 1.3567 \\
            64M  & 2.8224 & 1.8084 & 1.1439 & 1.0312 & 1.7689 & 2.5956 \\
            128M & 5.2902 & 3.7402 & 2.3268 & 1.8818 & 3.3816 & 5.7314 \\
            \hline
        \end{tabular}
    \end{center}

\section*{Part 2}
    \subsection*{I}
     The first 10 numbers after index 100,000:
      \begin{center}
        \begin{tabular}{ c  c  c  c  c }
            222352 & 222354 & 222355 & 222356 & 222358 \\ 
            222359 & 222361 & 222362 & 222365 & 222367
        \end{tabular}
      \end{center}

    \subsection*{I}
     The first 10 numbers after index 200,000:
      \begin{center}
        \begin{tabular}{ c  c  c  c  c }
            479822 &  479824 &  479835 &  479837 &  479841\\
            479845 &  479847 &  479848  & 479852 &  479853
        \end{tabular}
      \end{center}

\section*{Part 3}
    \begin{eqnarray}
        T_{bitonic} & = & \frac{N}{P} lg(\frac{N}{P}) + 2 \frac{N}{P} (1 + 2 + ...+ log P)\\
                    & = & \frac{N}{P} (lg(\frac{N}{P}) + 2 \frac{lg P ( 1 + lg P)}{2}) \\
                    & = & \frac{N}{P} ( lg N - lg P + lg P + lg^2 P)\\
                    & = & \frac{N}{P} (lg N + lg^2 P)
    \end{eqnarray}

    \begin{eqnarray}
        Speed\ Up & = & \frac{N lg N}{T_{bitonic}} \\
                  & = & \frac{P lg N}{lg N + lg^2 P}
    \end{eqnarray}

    \begin{eqnarray}
        Efficiency & = & \frac{Speed\ Up}{P}\\
                   & = & \frac{lg N}{lg N + lg^2 P}
    \end{eqnarray}

    When $N=P$:
    \begin{eqnarray}
        Speed\ Up & = & \frac{N}{1 + lg N}
    \end{eqnarray}

    \begin{eqnarray}
        Efficiency & = & \frac{1}{1 + lg N}
    \end{eqnarray}

    When $N >> P$:
    \begin{eqnarray}
        Speed\ Up & = & \frac{P lg N}{lg N + 1}
    \end{eqnarray}

    \begin{eqnarray}
        Efficiency & = & \frac{lg N}{lg N + 1}
    \end{eqnarray}

\section*{Part 4}
   \begin{samepage}
    \begin{center}
        Speed Up

        \begin{tabular}{| c | c | c | c | c | c | c |}
            \hline
            ~ & \multicolumn{6}{c|}{Number of Processors}\\
            \hline
            Size & 1 & 2 & 4 & 8 & 16 & 32\\
            \hline
            16M  & 1.00 & 1.88 & 2.36 & 2.70 & 1.83 & 1.12\\
            32M  & 1.00 & 1.66 & 2.54 & 3.26 & 1.81 & 1.05\\
            64M  & 1.00 & 1.56 & 2.47 & 2.73 & 1.60 & 1.09\\
            128M & 1.00 & 1.41 & 2.27 & 2.81 & 1.56 & 0.923\\
            \hline
        \end{tabular}
    \end{center}
   \end{samepage}

\newpage
   \begin{samepage}
    \begin{center}
        Efficiency

        \begin{tabular}{| c | c | c | c | c | c | c |}
            \hline
            ~ & \multicolumn{6}{c|}{Number of Processors}\\
            \hline
            Size & 1 & 2 & 4 & 8 & 16 & 32\\
            \hline
            16M  & 1.   &  .94 &  .59 & .3375 & .114375 & .035\\
            32M  & 1.   & .83  & .635 & .4075 & .113125 & .0328125\\
            64M  & 1.   & .78  & .6175& .34125& .1 & .0340625\\
            128M & 1. &  .705  & .5675& .35125& .0975& .02884375\\
            \hline
        \end{tabular}
    \end{center}
   \end{samepage}

    \subsection*{1}
    At first the speed up increases then it levels off as N increases,
    much like the way the theoretical equation does.
    \subsection*{2}
    The efficiency decreases as N increases, like the theoretical equation does.
    \subsection*{3}
    The efficiency decreases as P increases, like the theoretical equation does.

    All of the equations are consistent because the code was written correctly.
\end{document}
